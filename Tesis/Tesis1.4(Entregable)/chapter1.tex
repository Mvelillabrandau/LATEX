\chapter*{Introduccion}
\section{Motivación}

El nacimiento del internet en el año 1958 en EEUU a través de ARPA (\textit{Advanced Researchs Proyects Agency}) tuvo como objetivo la comunicación entre diversas universidades y las entidades militares con propósitos investigativos, la invención del telégrafo, el teléfono, la radio y el ordenador sentaron las bases para el desarrollo de esta nueva tecnología, la cual hoy en día ha revolucionado la informática y las comunicaciones como ninguna otra cosa, convirtiéndose en una herramienta de índole mundial, un mecanismo el cual nos permite diseminar información de manera inmediata , generando un medio de colaboración e interacción entre las personas y los ordenadores, desconociendo su ubicación física.\\

Durante los año 2000 - 2008 el uso diario del internet en las personas americanas se desplazaba entre un 60 y 70 por ciento[Digital Citize], este aumento en el uso de la internet, conlleva también a un aumento de la información presente en la web, dando como resultado el desarrollo de aplicaciones las cuales deben interactuar con un gran numero de usuarios y analizando una sobresaliente cantidad de información, motivo por el cual internet impulsado por las demandas de las aplicaciones cada vez más emergentes y las capacidades de las nuevas redes de comunicación, se ha convertido en un mosaico arquitectónico que resulta en una creciente complejidad y vulnerabilidades imprescindibles, entregando como resultado violaciones de capas, proliferación de subcapas, y la erosión del modelo de extremo a extremo, es bajo esta eventualidad que todos los cambios efectuados concluyen en un aumento de la complejidad, lo cual se traduce en una internet osificada, es por esta razón que los problemas anteriormente señalados no se encuentran directamente relacionados con los protocolos o mecanismo específicos del internet actual, mas bien son causados esencialmente por la incapacidad de integrar nuevos mecanismos, lo que quiere decir que los problemas son causados por la arquitectura de internet y podrian ser resueltos con un nuevo diseño de arquitectura de internet[2].\\

Como una futura propuesta de arquitectura de Internet, ICN (\textit{Information-Centric Networking}) también llamado \textit{Content Centric Networking} (CCN), pretende motivar la transición arquitectónica de la actual arquitectura centrada en el host a centrada en la información para difundir de manera eficiente y flexible la enorme información generada por una variedad de aplicaciones[1], convirtiéndose este en un enfoque que pretende ayudar a desarrollar la infraestructura del internet y así apoyar de manera directa el acceso a objetos de datos con nombre (NDO), a modo que lo anteriormente señalado es conseguido gracias a la característica clave del paradigma ICN, la presencia de memoria cache dentro de los nodos ICN, en donde cada uno de los contenidos es nombrado única e independientemente desde la ubicación del productor, facilitando el almacenamiento en cache en los nodos intermediarios. Así, por ejemplo, los consumidores solicitaran contenidos enviando el denominado paquete de interés, el cual lleva el nombre del contenido, mientras que el productor o cualquier nodo que mantenga una copia del contenido puede responder a la petición realizada.\\

A su vez, los usuarios ya antes mencionados como consumidores, realizan peticiones en la web, las cuales siguen una conducta dinámica caracterizándose por un elevado sesgo entre los diferentes conjuntos de peticiones, en otras palabras, dentro del universo de peticiones generadas existen conjuntos de peticiones que son regularmente solicitadas por los consumidores en intervalos de tiempos distintos, por lo contrario, otras peticiones escasamente son solicitadas. Dicho lo anterior, también existen situaciones dentro de un intervalo acotado de tiempo, donde surgen explosivamente peticiones las que se caracterizan por poseer un contenido en común, las cuales nacen por el desarrollo de un evento de interés popular teniendo como resultado un aumento sustancial en la demanda generada a los nodos, teniendo como consecuencia latencia y cogestion en las redes.


\section{Desafios}
%*******************************************************
Los desafíos que se afrontaran para la realización de proyecto de titulo I, son inicialmente el diseño de una nueva arquitectura de memoria caché para los nodos de las redes \textit{ICN} (\textit{Information Centric Network}), considerando el comportamiento de los usuarios por medio del tráfico de red. En segundo lugar, se debe incorporar una estrategia de caché(políticas de admisión, desalojo y reemplazo) y que será implementada en un simulador denominado \textit{ndnSIM} (\textit{Named Data Networking Simulator}), todo esto con el fin de mejorar resultados(Queresultados) en comparación con otras arquitecturas caché que se encuentran por defecto dentro del simulador(LFU, FIFO).\\

\section{Contribución de la tesis}
El aporte entregado por el proyecto de titulo, se puede identificar inicialmente por la creación diseño de una nueva arquitectura de memoria caché para los nodos de las redes \textit{ICN} (\textit{Information Centric Network}), considerando el comportamiento de los usuarios por medio del tráfico de red. En segundo lugar, se debe incorporar una estrategia de caché(políticas de admisión, desalojo y reemplazo) y que será implementada en un simulador denominado \textit{ndnSIM} (\textit{Named Data Networking Simulator}), todo esto con el fin de mejorar resultados de eficiencia de la memoria caché (Hit's, Miss) en comparación con otras arquitecturas caché que se encuentran por defecto dentro del simulador(LFU, FIFO). No obstante, acontinuacion se detallaran las contribuciones efectuadas por este trabajo:\\
\begin{itemize}
	\item Diseño de una nueva arquitectura cache bajo el paradigma de las redes ICN, que contenga una subdivision de tres grupos capaces de retener diferentes tipos de peticiones(intereses) de modo que la eficiencia del nodo se vea afectada positivamente. En cuanto a la division de la memoria caché, el primer segmento se encarga del almacenamiento de peticiones del tipo rafaga, la segunda de guardar las de tipo permanente y para la ultima seccion, la recaudacion de peticiones de tipo variables.\\
	\item Creación de una topologia de redes ICN, utilizando el simulador ndnSIM, la cual contenga dentro de sus nodos la arquitectura caché anteriormente señalada con la finalidad de inyectar trafico de datos en base al comportamiento usuario para la obtencion de resultados empiricos respecto a la nueva arquitectura caché diseñada.\\
\end{itemize}


\section{Estructura de la tesis}
A continuacion se realizara una breve reseña de como se estructura el siguiente trabajo:\\

El capitulo 2 abarca la descripcion del problema, en el se define el objetivo general, especificos, hipotesis y los alcances que contiene el proyecto de titulo. Incluyendo tambien la descripcion de la metodolgia de trabajo utilizada para el progreso del trabajo.\\

El capitulo 3 posee el marco teorico, en el cual se definen diferentes conceptos escensiales para el entedimiento del proyecto del titulo. Dicho lo anterior se encuentran los siguiente conceptos: Redes centradas en la información (ICN), ndnSIM, aplicaciones web de gran escala, estructura y politicas de remplazo del caché existentes y finalizando con el comportamiento del usuario de manera que se 

